\documentclass[11pt]{article}
\usepackage[UTF8]{ctex}
\usepackage[a4paper]{geometry}
\geometry{left=2.0cm,right=2.0cm,top=2.5cm,bottom=2.5cm}

\usepackage{comment}
\usepackage{booktabs}
\usepackage{graphicx}
\usepackage{diagbox}
\usepackage{amsmath,amsfonts,graphicx,amssymb,bm,amsthm}
\usepackage{algorithm,algorithmicx}
\usepackage[noend]{algpseudocode}
\usepackage{fancyhdr}
\usepackage{tikz}
\usepackage{graphicx}
\usepackage{verbatim}
\usetikzlibrary{arrows,automata}
\usepackage{hyperref}

\setlength{\headheight}{14pt}
\setlength{\parindent}{0 in}

\newtheorem{theorem}{Theorem}
\newtheorem{lemma}[theorem]{Lemma}
\newtheorem{proposition}[theorem]{Proposition}
\newtheorem{claim}[theorem]{Claim}
\newtheorem{corollary}[theorem]{Corollary}
\newtheorem{definition}[theorem]{Definition}


\newcommand\E{\mathbb{E}}
\newcommand{\hwid}{5}			% 第几次作业
\newcommand{\name}{周雨扬} 		% 你的名字
\newcommand{\id}{2000013061} 	% 你的学号


\usetikzlibrary{positioning}

\begin{document}

    \pagestyle{fancy}
    \lhead{Peking University}
    \chead{}
    \rhead{Operating Systems}

    \begin{center}
        {\LARGE \bf Homework \#\hwid}\\
        {\Large \name}\\
        {\Large \id}\\
    \end{center}

	\section{Challanges}
		\par 本次作业完成了所有的代码补全任务,做了如下 challange.
		
		\begin{itemize}
			\item The block cache has no eviction policy. Once a block gets faulted in to it, it never gets removed and will remain in memory forevermore. Add eviction to the buffer cache. Using the \texttt{PTE\_A} "accessed" bits in the page tables, which the hardware sets on any access to a page, you can track approximate usage of disk blocks without the need to modify every place in the code that accesses the disk map region. Be careful with dirty blocks.
		\end{itemize}
		
	\section{Exercise 1}
	
		
	\subsection*{env\_create()}
		\par 根据代码中给出的描述简单模拟即可。只需要找到 IO 权限对应的位即可。
		
		\par 最后在 \texttt{inc/mmu.h} 中找到了用于 IO 权限的位 \texttt{FL\_IOPL\_MASK}。之后直接模拟即可,没有细节。
		
	\subsection*{Question 1}
	
		\par 实际上并不需要。切换的时候 \texttt{env\_pop\_tf} 实质上已经保存了包括 IO 权限信息在内的数据。
		
	\section{Exercise 2}
	
	\subsection*{bc\_pgfault()}
	
		\par 首先我们需要在内存上声明一个页,用于存储从磁盘中提取上来的数据。该页的权限需要同时设置为 \texttt{P,U,W} 使得用户有权限其上的内容。注意声明页的视乎首先需要将地址按照页取整。
		
		\par 之后根据提示观察 \texttt{fs/ide.c}, 发现其中有函数 \texttt{ide\_read} 可以支持从磁盘上读入信息。由于我们读入的信息总是一页/一块,而磁盘管理单元是段,将其转换成段数,段编号后调用即可。
		
	\subsection*{flush\_block()}
	
		\par 首先判断其是否有更新到磁盘的必要。如果当前地址所在的页没有被映射,亦或者没有被修改(\texttt{PTE\_D} 为假),则没有必要进行修改。
		
		\par 之后我们需要将内存上的信息写入磁盘。类似于 \texttt{bc\_pgfault()}, 我们发现 \texttt{fs/ide.c} 中有函数 \texttt{ide\_write} 可以支持向磁盘上写入信息。调用参数设置也非常类似,略去。
		
		\par 最后我们将当前页上的 \texttt{PTE\_D} 位抹除即可。将权限位和 \texttt{PTE\_SYSCALL} 取交即可。
		
	\section{Exercise 3}
	
	\subsection*{alloc\_block()}
	
	\par 我们观察 \texttt{free\_block}。观察发现在释放块的时候,他将 \texttt{bitmap} 的某一位赋值成 0。据此我们可以推断:\texttt{bitmap} 是一个用类似于 \texttt{bitset} 的结构存储块是否被使用的数组,没有被使用当且仅当这一位值为 $1$. 这一推论也可以从 \texttt{block\_is\_free} 证实。
	
	\par 据此我们可以直接枚举所有块,如果其没有被使用的将其赋值为使用。最后我们需要强制将 \texttt{bitmap} 更新到磁盘中,否则会引起信息不同步,重新加载时候会出错。
	
	\section{Exercise 4}
	
	\subsection*{file\_block\_walk()}
	
	\par 观察 \texttt{inc/fs.h} 中的代码,发现至多有 \texttt{NDIRECT} 个直接访问的块,据此结合注释可以直接处理块编号小于 \texttt{NDIRECT} 或者块编号大于 \texttt{NINDIRECT + NDIRECT} 的情况。
	
	\par 否则我们需要先判断该文件的 \texttt{indirect} 部分有没有声明。如果没有声明且不允许声明则返回错误。否则我们可以通过 \texttt{alloc\_block} 声明一个块存储 \texttt{indirect} 块的信息,并将其初始全部赋值为 $0$.
	
	\par 最后类似于直接块的寻址方式寻址即可。
	
	\subsection*{file\_get\_block()}
	
	\par 类似的,如果块编号大于 \texttt{NINDIRECT + NDIRECT} 的情况,可以直接返回越界。否则我们可以直接使用 \texttt{file\_block\_walk()} 获取存储页编号的位置。
	
	\par 如果这个位置已经被赋值了,则可以直接返回。否则我们需要给这位置赋值一个全新块的编号。赋值方式类似于 \texttt{file\_block\_walk()},此处略去细节。
	
	\subsection*{Note}
	
	\par 如果 \texttt{alloc\_block()} 了一个编号为 0 的块,当前的实现其会被当成没有连接块。但是实际上它的确被赋值了。观察后文发现系统全部把 0 当成了没有赋值的标记。
	
	\par 这可能带来磁盘泄露的问题。一个解决方案是强制分配的时候重编号是的没有编号为 0 的块,另一个解决方案是修改判定没有块编号存储的条件。两者要修改的地方都很多 QAQ。
	
	\section{Exercise 5}
	
	\subsection*{serve\_read()}
	
	\par 参考上面的 \texttt{serve\_set\_size()}, 我们可以用相同代码获取已经打开文件的信息。之后我们观察注释,调用 \texttt{file\_read} 完成即可。注意操作的参数在 \texttt{req} 中,返回值存储在 \texttt{ret} 中。
	
	\par 最后记得维护 \texttt{offset},其记录了已经读取的长度。同时记得将读入的长度限制在 \texttt{BLKSIZE},否则最后一个练习会发生问题。
	
	\section{Exercise 6}
	
	\subsection*{serve\_write()}
	
	\par 参考上面的 \texttt{serve\_read()} 即可。传除去调用 \texttt{file\_write()} 参数略有不同之外其余几乎全部相同。
	
	
	\subsection*{devfile\_write()}
	
	\par 可以通过参考上面的 \texttt{devfile\_read} 获得向磁盘写的操作细节。除去调用参数,调用顺序略有不同,其余基本一致。注意仍然需要保证每次写入的长度不超过 \texttt{BLKSIZE}.
	
	\par 之后我加上了一层循环,向文件中不断的写入信息直到写完亦或者是发生错误。这里我们需要维护已经写入的长度,指向没有写入信息的指针。维护细节不多此处略去。
	

	\section{Exercise 7}
	
	\subsection*{sys\_env\_set\_trap\_frame()}
		
		\par 注释里的三个要求依次为:
		
		\begin{itemize}
			\item \texttt{cs} 最低两位设置为 $3$.
			\item \texttt{eflags} 的 \texttt{FL\_IF} 设置为真.
			\item \texttt{eflags} 的 \texttt{FL\_IOPL\_MASK} 全部设置为 0.
		\end{itemize}
		
		\par 简单模拟即可。之后记得加入 \texttt{syscall()} 中。
		
	\subsection*{i386\_init()}
		\par 插入 \texttt{ENV\_CREATE(user\_spawnhello, ENV\_TYPE\_USER)} 用于代码测试。
	
	\section{Exercise 8}
		
	\subsection*{duppage()}
		
		\par 如果当前页权限位包含 \texttt{PTE\_SHARE}, 简单的将其映射到子环境中相同的地址即可。由于是共享,权限位设置为 \texttt{PTE\_SYSCALL} 即可。
		
	\subsection*{copy\_shared\_pages()}
		
		\par 判定一个页是否需要拷贝与 \texttt{fork()} 中的规则类似,除了我们需要在页权限位上额外检查是否包含 \texttt{PTE\_SHARE}。拷贝过程与 \texttt{duppage()} 类似,复制即可。
		
	\section{Exercise 9}
		
	\subsection*{trap\_dispatch()}
		
		\par 两个错误和 \texttt{IRQ\_OFFSET} 的偏移量分别为 \texttt{IRQ\_KBD} 和 \texttt{IRQ\_SERIAL}, 处理函数分别为 \texttt{kbd\_intr()} 和 \texttt{serial\_intr()}. 据此可以直接实现。处理完后由于仅是中断,返回即可。
		
	\section{Exercise 10}
	
	\subsection*{runcmd()}
		
		\par 直接抄下面处理输出重定向的代码即可。注意打开文件时候权限仅有 \texttt{O\_RDONLY}, 不需要创建文件。
		
	\section{Challange 2}
	
	\subsection*{alloc\_block()}
	
		\par 如果没有空余的块,我们需要去删除某些已经不再使用的块。 仍然忽略所有的块,并且忽略掉没有 \texttt{PTE\_P} 的块。
		
		\par 首先我们判断这个块是否为脏块。如果是的话需要先 \texttt{flush\_block()} 刷新后,删除 \texttt{PTE\_D} 位。(不能用 \texttt{PTE\_SYSCALL} 消除,会将 \texttt{PTE\_A} 一起删除!)
		
		\par 之后我们再判断是否有 \texttt{PTE\_A}。如果有则说明这个块已经不再使用。此时则可以用 \texttt{sys\_page\_unmap} 和 \texttt{free\_block} 来释放这个块。
		
		\par 如果上述过程中我们释放了至少一个块,则我们可以直接记录其中一个被释放的块的信息,将其返回;否则如果一个块都没有被释放,则直接返回错误。
		
	
			
\end{document}